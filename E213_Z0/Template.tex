\documentclass[11pt, english, fleqn, DIV=15, headinclude, BCOR=2cm]{scrreprt}

\usepackage[
    color,
    bibatend,
]{../../header}

\graphicspath{{_build/}{../Figures/}}

\usepackage{mathtools}

\hypersetup{
    pdftitle=
}

\usepackage{longtable}
\usepackage{subcaption}

\usepackage[all]{nowidow}

\subject{Lab report}
\title{Analysis of $Z_0$ decays}
\subtitle{Experiment K213 -- Universität Bonn}
\author{%
    Martin Ueding \\
    \small{\href{mailto:mu@martin-ueding.de}{mu@martin-ueding.de}}
    \and
    Lino Lemmer \\
    \small{\href{mailto:l2@uni-bonn.de}{l2@uni-bonn.de}}
}

\date{2016-03-02}

\publishers{Tutor: Peter Klassen}

\begin{document}

\maketitle

\begin{abstract}
\end{abstract}

\tableofcontents

\chapter*{Permission to upload}

I, Martin Ueding, would like to scan and upload this lab report with your
corrections to my website \href{http://martin-ueding.de}{martin-ueding.de}.
There, the original lab report as well as the reviewed one will be licensed
under the “\href{http://creativecommons.org/licenses/by-sa/4.0/}{Creative
Commons Attribution-ShareAlike 4.0 International License}”. Is that okay with
you?

Yes $\Box$ \hspace{2cm} No $\Box$

\chapter{Theory}

\begin{figure}
    \centering
    \includegraphics{radiative_interpolation}
    \caption{%
        Radiation corrections given in the experiment description on page~4. We
        have interpolated with quadratic splines.
    }
    \label{fig:radiative_interpolation}
\end{figure}

\chapter{Exercises}

There are a couple of exercises that are supposed to be done before the
experiment.

\section{Decay width}

We are to compute the decay width~$\Gamma$ of a $\mathrm Z^0$ particle into a
pair of various fermions. The top-quark is heavier than the gauge boson,
therefore it cannot decay into a pair of top-quarks. The Feynman diagram
corresponding to the decay is shown in Figure~\ref{fig:z0-decay}.

\begin{figure}
    \centering
    \includegraphics{z0-decay}
    \caption{%
        Decay of a $\mathrm Z^0$ gauge boson into a pair of fermions. Time is
        to the right.
    }
    \label{fig:z0-decay}
\end{figure}

First we compute the invariant matrix element~$\mathcal M$. Reading off the
Feynman diagram, we have
\begin{align*}
    \iup \mathcal M
    &= \epsilon^\mu \bar u^s(\four p) \frac{\iup g}{\cos(\theta_\text w)}
    \mat\gamma_\mu \del{g_\text v^f - g_\text a \mat\gamma_5} v^{s'}(\four k)
    \,,
    \intertext{%
        where we have used the rules given by \textcite[(D.56)]{romao/aqt}. We
        chose the center of mass system, as we always can for a single massive
        particle, and have $\four q = (\sqrt s, \vec 0)$. As the gauge boson
        does not have any three momentum direction, we can choose the
        polarization vector $\four \epsilon$ like we want. We chose it in the
        positive $x^3$-direction and have $\four \epsilon = (0, 0, 0, 1)$. With
        that we can simplify the Lorentz structure a little bit and obtain
    }
    &= \bar u^s(\four p) \frac{\iup g}{\cos(\theta_\text w)}
    \mat\gamma_3 \del{g_\text v^f - g_\text a \mat\gamma_5} v^{s'}(\four k)
    \,.
\end{align*}

For the decay width, which we are interested in, we need the modulus squared of
the matrix element. For that we need the following:
\begin{multline*}
    \sbr{\bar u^s(\four p) \mat\gamma_3 \del{g_\text v^f - g_\text a \mat\gamma_5}
    v^{s'}(\four k)}^\dagger \\
    =
    v^{s'}(\four k)^\dagger
    \del{g_\text v^f - g_\text a \mat\gamma_5^\dagger}
    \mat\gamma_3^\dagger
    \mat\gamma_0
    u^s(\four p) \\
    =
    \underbracket{v^{s'}(\four k)^\dagger
    \mat\gamma_0}
    \underbracket{\mat\gamma_0
        \del{g_\text v^f - g_\text a \mat\gamma_5^\dagger}
    \mat\gamma_0}
    \underbracket{\mat\gamma_0
        \mat\gamma_3^\dagger
    \mat\gamma_0}
    u^s(\four p) \\
    = \bar v^{s'}(\four k)
    \del{g_\text v^f + g_\text a \mat\gamma_5}
    \mat\gamma_3
    u^s(\four p) \\
    = \bar v^{s'}(\four k)
    \mat\gamma_3
    \del{g_\text v^f - g_\text a \mat\gamma_5}
    u^s(\four p) \,. \\
\end{multline*}

Now we can write the modulus squared of the matrix element
\begin{align*}
    |\mathcal M|^2
    &= \frac{g^2}{\cos(\theta_\text w)^2}
    \bar v^{s'}(\four k)
    \mat\gamma_3
    \del{g_\text v^f - g_\text a \mat\gamma_5}
    u^s(\four p)
    \bar u^s(\four p)
    \mat\gamma_3
    \del{g_\text v^f - g_\text a \mat\gamma_5}
    v^{s'}(\four k)
    \,. \\
    \intertext{%
        As the detector is not sensitive to the spin of the resulting fermions,
        we need to sum over all the final state spin configurations $s$ and
        $s'$. This will give us a fermion trace,
    }
    \sum_\text{spins} |\mathcal M|^2
    &=
    \frac{g^2}{\cos(\theta_\text w)^2}
    \tr\del{
        \four k
        \mat\gamma_3
        \del{g_\text v^f - g_\text a \mat\gamma_5}
        \four p
        \mat\gamma_3
        \del{g_\text v^f - g_\text a \mat\gamma_5}
    }
    \,, \\
    \intertext{%
        where we have neglected the masses of the fermions. The second
        parentheses with $\mat\gamma_5$ can be anticommuted twice and be
        collapsed with the first one. We have
    }
    &=
    \frac{g^2}{\cos(\theta_\text w)^2}
    \tr\del{
        \four k
        \mat\gamma_3
        \del{g_\text v^f - g_\text a \mat\gamma_5}^2
        \four p
        \mat\gamma_3
    }
    \,, \\
    \intertext{%
        where we can now simplify this. As an aside, we have
        \[
            \del{g_\text v^f - g_\text a \mat\gamma_5}^2
            = (g_\text v^f)^2 - (g_\text a)^2 - 2 g_\text v^f g_\text a
            \mat\gamma_5 \,.
        \]
        The square of $\mat\gamma_5$ is just the unit matrix in Dirac space.
        The term with $\mat\gamma_5$ will drop out as the trace identity
        contains the Levi-Civita symbol~$\epsilon_{\mu\nu\rho\gamma}$ which
        will be zero if two of the indices are 3, as we have in our case.
        Therefore we can directly drop that term. As it does not contain any
        further Dirac structure, we can pull it out front and obtain
    }
    &=
    \frac{g^2}{\cos(\theta_\text w)^2}
    \del{(g_\text v^f)^2 - (g_\text a)^2}
    \tr\del{
        \four k
        \mat\gamma_3
        \four p
        \mat\gamma_3
    }
    \,.
    \intertext{%
        Using a trace identity like given by \textcite[(A.27)]{Peskin/QFT/1995}
        we obtain
    }
    &= \frac{4 g^2}{\cos(\theta_\text w)^2} \del{(g_\text v^f)^2 - (g_\text a)^2}
    (2 k_3 p_3 + \four k \cdot \four p) \,.
\end{align*}
This is our final form of the matrix element.

The decay width is given by additional factors and a integration over phase
space. We have
\begin{align*}
    \Gamma_f
    &= \frac{1}{2 M_{\mathrm Z^0}} (2 \piup)^4 \int
    \frac{\dif^3 p}{(2\piup)^3 2 E_{\vec p}}
    \frac{\dif^3 k}{(2\piup)^3 2 E_{\vec k}}
    \delta^4(\four q - \four k - \four p)
    |\mathcal M|^2 \,.
    \intertext{%
        We simplify the factors and insert the matrix element and get
    }
    &= \frac{1}{32 \piup^2 M_{\mathrm Z^0}} \int
    \frac{\dif^3 p}{E_{\vec p}}
    \frac{\dif^3 k}{E_{\vec k}}
    \delta^4(\four q - \four k - \four p)
    \frac{4 g^2}{\cos(\theta_\text w)^2}
    \del{(g_\text v^f)^2 - (g_\text a)^2}
    (2 k_3 p_3 + \four k \cdot \four p) \,.
    \intertext{%
        Then we simplify even more and obtain
    }
    &= \frac{1}{8 \piup^2 M_{\mathrm Z^0}}
    \frac{g^2}{\cos(\theta_\text w)^2}
    \del{(g_\text v^f)^2 - (g_\text a)^2}
    \int
    \frac{\dif^3 p}{E_{\vec p}}
    \frac{\dif^3 k}{E_{\vec k}}
    \delta^4(\four q - \four k - \four p)
    (2 k_3 p_3 + \four k \cdot \four p) \,.
    \intertext{%
        The total energy-momentum conserving Dirac-distribution can be split
        into time-like and space-like part. In the center-of-mass frame, this
        gives us
    }
    &= \frac{1}{8 \piup^2 M_{\mathrm Z^0}}
    \frac{g^2}{\cos(\theta_\text w)^2}
    \del{(g_\text v^f)^2 - (g_\text a)^2}
    \\&\qquad\times
    \int
    \frac{\dif^3 p}{E_{\vec p}}
    \frac{\dif^3 k}{E_{\vec k}}
    \delta(\sqrt s - E_{\four k} - E_{\four p})
    \delta^3(\vec k + \vec p)
    (2 k_3 p_3 + \four k \cdot \four p) \,,
    \intertext{%
        which suggests the following change in variables:
        \[
            \vec a := \vec p + \vec k
            \eqnsep
            \vec b := \frac{\vec p - \vec k}{2} \,.
        \]
        The Jacobian of this transformation is unity. The $\delta^3(\vec k +
        \vec p)$ will become $\delta^3(\vec a)$, in the $\int \dif^3 a$
        integral, this will set $\vec a$ to $\vec 0$. So far we have
    }
    &= \frac{1}{8 \piup^2 M_{\mathrm Z^0}}
    \frac{g^2}{\cos(\theta_\text w)^2}
    \del{(g_\text v^f)^2 - (g_\text a)^2}
    \int
    \frac{\dif^3 b}{E_{\vec p} E_{\vec k}}
    \delta(\sqrt s - E_{\vec k} - E_{\vec p})
    (2 k_3 p_3 + \four k \cdot \four p) \,.
    \intertext{%
        Now we have to replace $k$ and $p$ with $b$. We have $\vec p = \vec b$
        and $\vec k = - \vec b$. Their magnitude is the same, which is expected
        in the center-of-mass frame in a two body decay. This simplifies the
        relations to
    }
    &= \frac{1}{8 \piup^2 M_{\mathrm Z^0}}
    \frac{g^2}{\cos(\theta_\text w)^2}
    \del{(g_\text v^f)^2 - (g_\text a)^2}
    \int
    \frac{\dif^3 b}{E_{\vec b}^2}
    \delta(\sqrt s - 2 E_{\vec b})
    (- 2 b_3 b_3 + E_{\vec b}^2 + \vec b \cdot \vec b) \,.
\end{align*}

%%%%%%%%%%%%%%%%%%%%%%%%%%%%%%%%%%%%%%%%%%%%%%%%%%%%%%%%%%%%%%%%%%%%%%%%%%%%%%%

\end{document}

% vim: spell spelllang=en_us tw=79
