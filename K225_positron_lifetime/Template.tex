\documentclass[11pt, english, fleqn, DIV=15, headinclude, BCOR=2cm]{scrreprt}

\usepackage[
    color,
    bibatend,
]{../../header}

\graphicspath{{./}{../Figures/}}

\newcommand\MZ{M_{\mathrm Z^0}}
\newcommand\electron{\mathrm e^-}
\newcommand\positron{\mathrm e^+}

\usepackage{needspace}

\usepackage{mathtools}
\usepackage{listings}

\lstset{
    basicstyle=\small\ttfamily,
}

\hypersetup{
    pdftitle=
}

\usepackage{longtable}
\usepackage{subcaption}

\usepackage[all]{nowidow}

\subject{Lab report}
\title{Positron lifetime in metals and insulators}
\subtitle{Experiment K225 -- Universität Bonn}
\author{%
    Martin Ueding \\
    \small{\href{mailto:mu@martin-ueding.de}{mu@martin-ueding.de}}
    \and
    Lino Lemmer \\
    \small{\href{mailto:l2@uni-bonn.de}{l2@uni-bonn.de}}
}

\date{\daterange{2016-03-24}{2016-03-25}}

\publishers{Tutor: Martin Urban}

\begin{document}

\maketitle

\begin{abstract}
        In this experiment we use a fast-slow-coincidence circuit to measure
        the temperature dependence of the positron lifetime in metals. Due to
        different lifetimes of free and trapped positrons we can examine the
        formation of vacancies in the metal. 
        
        In an overnight measurement we determine the lifetimes of free and
        trapped positronium in an insulator. 
\end{abstract}

\tableofcontents

\chapter*{Permission to upload}

I, Martin Ueding, would like to scan and upload this lab report with your
corrections to my website \href{http://martin-ueding.de}{martin-ueding.de}.
There, the original lab report as well as the reviewed one will be licensed
under the “\href{http://creativecommons.org/licenses/by-sa/4.0/}{Creative
Commons Attribution-ShareAlike 4.0 International License}”. Is that okay with
you?

Yes $\Box$ \hspace{2cm} No $\Box$

\chapter{Theory}

\section{Positrons}

\subsection{Positron sources}

Our source of positrons will be $\mathrm{^{22}Na}$ throughout the experiment.
Figure~\ref{fig:na22} shows that the sodium decays via a $\betaup^+$-decay or
electron capture into an excited neon isotope. Both decay mechanism are related
via the crossing symmetry and have the same amplitude. Figure~\ref{fig:beta}
shows the first variant. The excited neon will quickly decay into the ground
state and emit a photon. Said photon will mark the creation of the positron.

\begin{figure}
    \centering
    \includegraphics{na22}
    \caption{%
        Decay of $\mathrm{^{22}Na}$.
    }
    \label{fig:na22}
\end{figure}

Due to the creation of the neutrino, the energy spectrum of the positron is
continuous.

\begin{figure}
    \centering
    \includegraphics{beta}
    \caption{%
        Feynman diagram for the $\betaup^+$-decay.
    }
    \label{fig:beta}
\end{figure}

\subsection{Positron annihilation}
\label{ssec:pos_ann}

The tree-level annihilation diagrams are shown in
Figure~\ref{fig:annihilation}. Depending on the relative spin orientations of
the electron and positron, a decay into an even or odd number of photons is
possible. The decay into a single photon is not possible due to
energy-momentum-conservation. Also, the additional creation of two more photon
suppresses the cross section by $\alpha \approx 1/137$. Therefore we can assume
that only decays in two and three photons occur.

\begin{figure}
    \centering
    \begin{subfigure}[c]{0.48\linewidth}
        \centering
        \includegraphics{two-photon}
        \caption{%
            Two photons
        }
        \label{fig:/1}
    \end{subfigure}
    \hfill
    \begin{subfigure}[c]{0.48\linewidth}
        \centering
        \includegraphics{three-photon}
        \caption{%
            Three photons
        }
        \label{fig:/2}
    \end{subfigure}
    \caption{%
        Decay of a positron into photons.
    }
    \label{fig:annihilation}
\end{figure}

\subsection{Positronium formation}

A positron can form a hydrogen-like metastable bound state with an electron.
Depending on the particle's spins this state is called para-positronium (spins
antiparallel) or ortho-positronium (spins parallel). The positronium atom will
decay eventually due to a finite possibility for the positron to be located at
the electron's position and vice versa. As written in
Section~\ref{ssec:pos_ann} the para-positronium only decays in two photons and
the ortho-positronium only in three photons. If the latter is located near
other atoms, the long lifetime leads to the so called pick-off process, where
the positron annihilates with one of the atom's electrons with antiparallel
spin. This again leads to a reduced lifetime, which still is remarkably longer
than the para-positronium's one.

\section{Metals}

A perfect metal consists of only atoms of one type. They are arranged in a
perfect periodic lattice, usually a Bravais lattice with a trivial unit cell.
Some electrons can move rather freely between the atoms as their wave functions
are mostly delocalized in the conducting band.

\subsection{Lattice defects}

In reality, the perfect periodic lattice is not realized. There are other atoms
in the lattice, taking a regular lattice site. Those foreign atoms might be
larger or smaller than the regular atoms. The number of electrons can be
different for such a foreign atom. In doped semiconductors, this is
deliberately done to obtain certain electronic properties.

There are also mesoscopic lattice defects like shifts, tears or similar
disruptions to the periodicity. At the surfaces where such a shift happen,
the distance between the atoms becomes very irregular.

\subsection{Vacancies}

The lattice might be disturbed in a way which leaves gaps in the lattice, sites
are not filled with an atom. Those spots are called vacancies. There, electrons
neighboring atoms aggregate as there is no repelling positive charge.

The density of vacancies depends on the temperature. The higher the
temperature, the more vacancies we expect. At temperature $T$, the density of
vacancies is given by \textcite[(3)]{Weiler/Vacancy_formation} as
\[
        C(T) = \exp\del{\frac{S_\text{t}}{k}} \exp\del{-\frac{H_\mathrm t}{kT}} \,.
\]
$H_\mathrm t$ is the energy needed for a vacancy to form, it is called
\enquote{vacancy formation enthalpy}.

\subsection{Trapping model}
\label{ssec:tra_mod}

In a metal, the amount of free electrons is very large. A positron cannot form
positronium as the amount of competing electrons is too high. The positron can
still come to rest in a vacancy as the net negative charge there will attract
and trap it.

Our trapping model is simple: Positrons can either be free and move through the
metal (Bloch wave) or be trapped in one of the vacancies. The lifetime for both
states is different, a free positron will have more opportunity to annihilate
than one in a vacancy with less electron density.

A trapped positron could become free again. In our experiment we will not see
this as the positron will annihilate before becoming free again. The model can
therefore be described by a free lifetime, a trapping rate and a trapped
lifetime.

As a free positron can either annihilate or become trapped, its total lifetime
is
\[
    \tau_0 = \del{\frac{1}{\tau_\mathrm f} + \sigma_\mathrm t C_\mathrm t}\inv
\]
as given by \textcite[(1a)]{Weiler/Vacancy_formation}.

The trapped lifetime simply is $\tau_\mathrm t$.

In an insulator, there is no conducting band of electrons. Positrons can
actually find an electron and form positronium with it. 

\section{Measurement technique}

\subsection{LYSO scintillator}

In order to detect the photons, we use a \textsc{lyso} scintillator. This material is
radioactive itself and provides a built-in energy calibration. The decay scheme
of the radionuclide is shown in Figure~\ref{fig:scheme-176Lu}.

\begin{figure}
    \centering
    \includegraphics{scheme-176Lu}
    \caption{%
        Decay scheme of $^{176}\mathrm{Lu}$ into $^{176}\mathrm{Hf}$.
    }
    \label{fig:scheme-176Lu}
\end{figure}

Photons interact with matter through three main channels: photo effect, Compton
effect and pair production. In the case of the scintillation material, we are
not really interested in the details but use that usually the full energy of
the incident photon is deposited. The material will exhibit a light pulse which
is proportional to the incident energy.

\subsection{Photomultiplier}

The light pulse from the scintillator is too faint to measure directly. It is
therefore amplified with a photomultiplier. A schematic is shown in
Figure~\ref{fig:photomultiplier}. Incoming photons will free an electron via
the photo effect. The electron is then accelerated via several electric fields.
Once it hits the first dynode, the accelerated electron has enough energy to
create an avalanche of secondary electrons. This process repeats along the
dozen dynodes until a measurable current is created.

\begin{figure}
    \centering
    \includegraphics{photomultiplier}
    \caption{%
        Schematic construction of a photomultiplier.
    }
    \label{fig:photomultiplier}
\end{figure}

The signal at the very end of the multiplier might be saturated and have lost
its energy proportionality. This signal will rise very fast due to quick
saturation, therefore it is also called \enquote{fast signal}. Picking up the
signal from one of the last dynodes will give a slower rising but
energy-proportional signal, the \enquote{slow signal}.

\subsection{Single channel analyzer (SCA)}

We want to measure the lifetime of the positron by measuring the duration
between the $\betaup^+$-decay and the annihilation of the positron. Therefore we
need to filter out the \SI{1275}{\kilo\electronvolt} and
\SI{511}{\kilo\electronvolt} lines. A single channel analyzer takes an analog,
energy proportional signal and gives a digital pulse if the amplitude lies in a
certain interval. This interval is also called \enquote{\textsc{sca} window}. Models
either have an upper and lower limit or a lower limit and a window width.

\subsection{Time-to-amplitude converter (TAC)}

In order to measure the durations in the order of nanoseconds, we use a start
and stop signal to a \textsc{tac}\@. This device starts charging a capacitor with a
constant current. Once the stop-signal is given, the capacitor is discharged,
giving a duration proportional amplitude.

\subsection{Multi channel analyzer (MCA)}

We will feed the \textsc{tac} amplitude into a multi channel analyzer. This is a
hardware device that creates a histogram of incoming analog pulses. In our case
this is realized as a PCI-card in a computer and proprietary software running
on Windows XP only.

\subsection{Fast-slow coincidence circuit}

The whole setup to measure the positron lifetime is the \enquote{fast-slow
coincidence circuit}. It is shown in Figure~\ref{fig:fast-slow}. The setup will
measure the duration between a \SI{1275}{\kilo\electronvolt} (start) photon in
the upper detector and a \SI{511}{\kilo\electronvolt} (stop) photon in the
lower detector.

\begin{figure}
    \centering
    \includegraphics{fast-slow}
    \caption{%
        Schematic of the fast-slow circuit. Wavy lines are photons, solid lines
        analog signal and dash-dotted lines denote digital pulses.
    }
    \label{fig:fast-slow}
\end{figure}

The start photon will be detected in the upper \textsc{lyso} detector and amplified by
the photomultiplier (PM). The signal from the last dynode is amplified and fed
into the upper \textsc{sca}\@. If the amplitude (i.e.\ the photon energy) matches, the \textsc{sca}
emits a long rectangular pulse to the coincidence unit. The fast signal from
the same photomultiplier is given to a \textsc{cfd}\@. This upper \textsc{cfd} will trigger the
start on the \textsc{tac}\@.

A short time later, when the positron has decayed, a
\SI{511}{\kilo\electronvolt} photon will reach the other \textsc{lyso} detector. The
same outputs are used and fed to the lower \textsc{sca} and \textsc{cfd}, respectively.
Overlapping with the pulse of the upper \textsc{sca}, the lower \textsc{sca} will emit a
rectangular pulse as well. Both pulses match in the coincidence unit and open
the gate on the \textsc{mca}\@. The output of the lower \textsc{cfd} is delayed to make sure that
it arrives later at the \textsc{tac}\@.

Once the stop signal arrived at the \textsc{tac}, the latter will emit an analog pulse
to the \textsc{mca} which is proportional to the duration between start and stop. In the
\textsc{mca} a histogram of durations is therefore built up.

\subsection{Prompt curve}

The bins in the \textsc{mca} histogram do not have any time scale associated with them,
a time gauge needs to be derived first. For this we use that electrons and
positrons decay into two back-to-back photons when their net spin is zero. We
can set both \textsc{sca} windows to the typical \SI{511}{\kilo\electronvolt} energy and
assume that both photons are emitted simultaneously. We expect to measure a
duration of \SI{0}{\nano\second} plus the explicit delay. On the \textsc{mca} this
should give a sharp peak which marks our zero. Adding more time to the delay
will shift this peak in the histogram. The distance of the peaks will give us
the relation between duration and histogram bins. This process will be done
several times for statistical accuracy.

\section{Analysis models}

\subsection{Prompt curves}

Due to the limited time resolution of the whole setup the peaks are broadened.
We have to fit a resolution function to the data:
\[
P(t) = \frac1{\sqrt{2\piup}\sigma}
        \exp\del{-\frac{\del{t-t_0}^2}{2\sigma^2}}.
\]

\subsection{Lifetime spectrum}

As in Section~\ref{ssec:tra_mod} stated the lifetime spectrum contains a
superposition of two decays with different lifetimes $\tau_0$ and
$\tau_\text{t}$ and their corresponding relative intensities $I_0$ and
$I_\text{t}$:
\begin{align*}
    W (t) &= \frac{I_0}{\tau_0}\eup^{-\frac{t}{\tau_0}} +
    \frac{I_\text{t}}{\tau_\text{t}}\eup^{-\frac{t}{\tau_\text{t}}}.
    \intertext{%
        For the detecting system has a limited time resolution the
        obtained spectrum is a convolution of the real lifetime
        spectrum with the resolution function:
    }
    M (t) &= (W * P)(t) \\
          &= \sum_{i\in\{0,\text{t}\}}\frac{I_0}{2\tau_i}
    \exp\del{\frac{\sigma^2-2\tau_i(t-t_0)}{2\tau_i^2}}
    \del{\text{erf}\del{\frac{\sigma^2 + \tau_it_0} {\sqrt2\sigma\tau_i}}
    + \text{erf}\del{\frac{\tau_i(t-t_0)-\sigma^2}{\sqrt2\sigma\tau_i}}},
\end{align*}
with $\text{erf}(x)$ being the Gaussian error function. For
fitting data, where we have a background $N_\text{BG}$, we use
\begin{equation}
    \label{eq:lifetime-fit}
    \begin{aligned}
        N (t) &= \sum_{i\in\{0,\text{t}\}}\frac{A_i}{2\tau_i}
        \exp\del{\frac{\sigma^2-2\tau_i(t-t_0)}{2\tau_i^2}}
        \del{\text{erf}\del{\frac{\sigma^2 + \tau_it_0} {\sqrt2\sigma\tau_i}}
        + \text{erf}\del{\frac{\tau_i(t-t_0)-\sigma^2}{\sqrt2\sigma\tau_i}}}
        \\
        &\qquad + N_\text{BG} \,.
    \end{aligned}
\end{equation}
The count amplitudes $A_0$ and $A_\text{t}$ are related to the
intensities by
\[
    I_i = \frac{A_i}{A_0 + A_\text{t}} \,.
\]
With this we obtain the mean lifetime as
\[
    \bar\tau = I_0\tau_0 + I_\text{t}\tau_\text{t} \,.
\]

\subsection{Temperature dependence}

The mean lifetime shows a sigmoidal temperature dependence, as given in the
additional information for this experiment:
\begin{align}
    \nonumber
    \bar\tau &= \tau_\text{f} \frac{1 + \sigma C(T) \tau_\text{t}}{1 +
    \sigma C(T) \tau_\text{f}} \\
    \label{eq:temp-model}
    &= \tau_\text{f} \frac{1 + \sigma \exp\del{\frac{S_\text{t}}{k}}
    \exp\del{-\frac{H_\mathrm t}{kT}} \tau_\text{t}} {1 + \sigma
    \exp\del{\frac{S_\text{t}}{k}} \exp\del{-\frac{H_\mathrm t}{kT}}
\tau_\text{f}} \,.
\end{align}
This shows, that the mean lifetime for low temperatures is
approximately $\tau_\text{f}$. For high temperatures it reaches
$\tau_\text{t}$, due to more vacancies. Fitting this to the
mean lifetime allows to extract for example $\tau_\text{f}$. 
Another way to get $\tau_\text{f}$ is given by
\textcite[(6)]{Weiler/Vacancy_formation}:
\[
    \tau_\text{f} = \del{\frac{I_0}{\tau_0} +
    \frac{I_\text{t}}{\tau_\text{t}}}^{-1} \,.
\]
The trapping rate $\sigma C(T)$ can be written as
\[
    \sigma C(T) = \frac{\bar\tau - \tau_\text{f}}
    {\tau_\text{f}(\tau_\text{t}-\bar\tau)} \,.
\]

By plotting $\log(\sigma C)$ against $1/T$ in an Arrhenius
diagram and fitting a linear function to the data, one can
derive the vacancy formation enthalpy $H_\text{t}$ from the
slope:
\begin{equation}
    \label{eq:arrhenius}
    \log\del{\frac{\bar\tau - \tau_\text{f}}
    {\tau_\text{f}(\tau_\text{t}-\bar\tau)}} = \log(\sigma) +
    \frac{S_\text{t}}k - \frac{H_\text{t}}{k}\cdot\frac1T \,.
\end{equation}


\chapter{Procedure and Analysis}

We want to measure various lifetimes and related quantities using a fast-slow
coincident circuit. Therefore we need to start constructing it from the basic
building blocks and make sure that they all work as desired.

\section{Slow circuit setup}

First we set up the slow circuit part in several stages.

\subsection{LYSO spectrum}

We have to associate the detected events with the corresponding transitions
before adjusting the amplifiers. To separate the decay of the
${}^{22}\text{Na}$ from the intrinsic \textsc{lyso} spectrum, we first take the
\textsc{lyso} spectrum of the left and right detectors. In
Figure~\ref{fig:lyso} one can clearly determine the
$\SI{596.82}{\kilo\electronvolt}$ cascade decay of the $6^+$ state of
${}^{176}\text{Hf}$.

\begin{figure}
        \centering
        \begin{subfigure}[c]{.49\linewidth}
                \centering
                \includegraphics{lyso-li}
                \caption{%
                        Left detector
                }
                \label{fig:lyso-li}
        \end{subfigure}
        \hfill
        \begin{subfigure}[c]{.49\linewidth}
                \centering
                \includegraphics{lyso-re}
                \caption{%
                        Right detector
                }
                \label{fig:lyso-re}
        \end{subfigure}
        \caption{%
                Intrinsic energy spectrum of the \textsc{lyso} detector.
        }
        \label{fig:lyso}
\end{figure}

\subsection{Gain adjust using sodium}

In the sodium spectrum, we expect to see the \SI{511}{\kilo\electronvolt} line
from the annihilation as well as a \SI{1275}{\kilo\electronvolt} line from the
excited neon.

We adjust the gain of the amplifiers such that the sodium spectrum fill the
8192 channels of the \textsc{mca}. The maximum window width of the \textsc{sca}
is not sufficient to let everything through. Also the upper bound for the
\textsc{sca} is below the last channel of the \textsc{mca}. Therefore we cannot
set the amplifier as high as we would like to. For the spectrum, we adjust the
lower bound such that the window covers the two lines we are interested. The
spectra of the left and the right detector are shown in
Figure~\ref{fig:natrium}.

\begin{figure}
        \centering
        \begin{subfigure}[c]{.49\linewidth}
                \centering
                \includegraphics{na-li}
                \caption{%
                        Left detector.
                }
                \label{fig:na-li}
        \end{subfigure}
        \hfill
        \begin{subfigure}[c]{.49\linewidth}
                \centering
                \includegraphics{na-re}
                \caption{%
                        Right detector.
                }
                \label{fig:na-re}
        \end{subfigure}
        \caption{%
                Spectrum of ${}^{22}\text{Na}$. The lines of the detector
                material are also visible. The trigger is set to the rising
                flank of one of the signals.
        }
        \label{fig:natrium}
\end{figure}

\subsection{Coincidence verification}

For the next few steps, we need simultaneous events from the left and right
detectors. Our aim is to make sure that coincident events show up as such in
the coincidence unit. The manual says to use the intrinsic spectrum from the
\textsc{lyso} detectors. Our tutor suggested to use the back-to-back
annihilation of positrons into two photons instead.

We set both left and right \textsc{sca} to give the
\SI{511}{\kilo\electronvolt} line. We check the coincidence of both
\textsc{sca} signals, the oscillogram is shown in
Figure~\ref{fig:1-sca-coincidence-511}. One can see that they overlap very
nicely.

\begin{figure}
    \centering
    \includegraphics[width=.48\linewidth]{br-1-sca-coincidence-511}
    \caption{%
        \textsc{sca} coincidence checked with the electron-positron
        annihilation back-to-back radiation. The oscilloscope only had a floppy
        disk interface, therefore we had to take pictures of the oscillogram
        and invert them.
    }
    \label{fig:1-sca-coincidence-511}
\end{figure}

Then we also took a quick look at the slow signal itself. For this we fed the
slow signal and the matching \textsc{sca} signal into the oscilloscope,
triggering onto the \textsc{sca} signal. See Figure~\ref{fig:2-sca-and-slow}
for the oscillogram.

% TODO Why is the SCA signal before the slow signal? That does not really make
% any sense. Was there some additional delay?

\begin{figure}
    \centering
    \begin{subfigure}[c]{0.48\linewidth}
        \centering
        \includegraphics[width=\linewidth]{br-2-sca-and-slow}
        \caption{%
            \textsc{sca}
        }
        \label{fig:2-sca-and-slow}
    \end{subfigure}
    \hfill
    \begin{subfigure}[c]{0.48\linewidth}
        \centering
        \includegraphics[width=\linewidth]{br-3-cfd-and-slow}
        \caption{%
            \textsc{cfd}
        }
        \label{fig:3-cfd-and-slow}
    \end{subfigure}
    \caption{%
        Slow signal with \textsc{sca} and \textsc{cdf}. Always triggered on the
        digital pulse
    }
    \label{fig:}
\end{figure}

\section{Fast circuit setup}

One slow signal and the corresponding \textsc{cfd} signal are fed into the
oscilloscope. We trigger onto the \textsc{cfd} dip. In the second channel
(orange), we can see various amplitudes. We want to remove the zero line by
setting the threshold as high as we need. Once the zero line vanishes in slow
signal (see Figure~\ref{fig:3-cfd-and-slow}), the \textsc{cfd} is set up. We
repeat this process with the other side.

Having both \textsc{cfd}s set up, we can look at both output at the same time.
The signal from the stop branch is led through a delay. With the help of the
oscillogram (Figure~\ref{fig:4-cfd-delay}), we set that delay to
\SI{20}{\milli\second}. This delay will produce an overall shift in our
lifetime curves. As they are only interested in the lifetime, the absolute time
if not of interest to us. Therefore we can set a delay time to make sure that
start and stop signal are well separated.

\begin{figure}
    \centering
    \begin{subfigure}[c]{0.48\linewidth}
        \centering
        \includegraphics[width=\linewidth]{br-4-cfd-delay}
        \caption{%
            Delay between \textsc{cfd}s.
        }
        \label{fig:4-cfd-delay}
    \end{subfigure}
    \hfill
    \begin{subfigure}[c]{0.48\linewidth}
        \centering
        \includegraphics[width=\linewidth]{br-5-tac-511}
        \caption{%
            Signal from the \textsc{tac}.
        }
        \label{fig:5-tac-511}
    \end{subfigure}
    \caption{%
        Oscillograms of time duration measurement with \textsc{cfd} and
        \textsc{tac}.
    }
    \label{fig:}
\end{figure}

The output of both \textsc{cfs}s (with delay in stop branch) are hooked up to
the \textsc{tac}. We make sure to use the same cables as before in order to
keep the delay by the cables constant. The \textsc{tac} produces an output
which we interpret as jitter around the fixed delay of \SI{20}{\milli\second}.
The physical process that we observe are the back-to-back photons from the
positron annihilation. Figure~\ref{fig:5-tac-511} shows the oscillogram
generated by the \textsc{tac}.

The fast circuit is now prepared for the rest of the experiment.

\section{Time calibration}

The coincidence of the two \textsc{sca}s has already been checked. Now we have
to make sure that the \textsc{tac} and the coincidence unit have sufficient
overlap. We will use the coincidence unit as the gate for the \textsc{mca}. If
there was insufficient overlap, the \textsc{mca} would not record the analog
spectrum from the \textsc{tac}.

In Figure~\ref{fig:6-tac-and-coincidence-511} we can see the digital pulse from
the coincidence unit is very long and completely encompasses the analog output
of the \textsc{tac}.

\begin{figure}
    \centering
    \includegraphics[width=.48\linewidth]{br-6-tac-and-coincidence-511}
    \caption{%
        Overlap of analog \textsc{tac} signal (orange) and digital coincidence
        signal (blue).
    }
    \label{fig:6-tac-and-coincidence-511}
\end{figure}

For a time calibration of our \textsc{mca}, we use the coincidence unit as
\textsc{mca}-gate and the \textsc{tac} as input. We take six prompt curves with
that setup. After each measurements, another \SI{4}{\nano\second}-delay is
interposed. For the first five prompt curves an acquisition time of about
\SI{2}{\minute} is used. They are shown together with their fitted Gauss curves
in Figure~\ref{fig:prompts_short}. The sixth prompt curve and its fit is shown
in Figure~\ref{fig:prompts_long}. The time resolution the acquisition time is
\SI{20}{\minute}, we want to get enough data to estimate the time resolution of
the circuit.

\begin{figure}
    \centering
    \includegraphics{prompts_short}
    \caption{%
        Prompt curves for time calibration. Acquisition time for every
        curve is approximately \SI{2}{\minute}.
    }
    \label{fig:prompts_short}
\end{figure}

\begin{figure}
    \centering
    \includegraphics{prompts_long}
    \caption{%
        Additional prompt curve for time calibration and time
        resolution estimation. Acquisition time is approximately
        \SI{20}{\minute}.
    }
    \label{fig:prompts_long}
\end{figure}

From the fitted functions we get the mean channel to every delay time. The
results are in Table~\ref{tab:time_gauge_param}. The error estimation in the
curve fitting procedure is performed using resampling pseudo-bootstrap. If we
would have a lot of time, we could have repeated each prompt curve measurement
a lot of times, say a hundred times. Each measurement would roughly give the
same curve. Statistical fluctuations would give us slightly different counts in
each bins.

\begin{table}
        \centering
        \begin{tabular}{SS}
                \toprule
                {delay time/\si{\nano\second}}
                & {mean channel number} \\
                \midrule
                %< for row in time_gauge_param >%
                << ' & '.join(row) >> \\
                %< endfor >%
                \bottomrule
        \end{tabular}
        \caption{%
                Mean channel number of Gauss fit width corresponding delay
                times.
        }
        \label{tab:time_gauge_param}
\end{table}

Taking that many measurements is unrealistic as we only have one lab day for
this. We need to generate more measurements from the ones that we have. This is
the origin of the name \enquote{bootstrap}, we use the data that we have to
generate more. This is as outrageous as pulling oneself up on one's bootstraps.
The rational is the following. We have a counting experiment with $n_i$ counts
in the $i$-th bin. Generally, a statistical error of $\sqrt{n_i}$ can be
associated with that count. Assuming that the implied distribution is Gaussian,
we can generate more points from a Gaussian distribution with mean $\mu = n_i$
and standard deviation $\sigma = \sqrt{n_i}$. Performing this for all bins, we
obtain a new \emph{bootstrap sample}.

The analysis is performed for each of those samples as usual. In our case it is
the curve fitting of a Gaussian distribution to the data. From out
\texttt{scipy.optimize.curve\_fit} we get optimal parameters for the fitted
model. For each of the fits onto the bootstrap samples, we record those
parameters. Then we take mean and standard deviation to obtain a value and its
error estimate.

In contrast to Gaussian error propagation, we did not have to take any
derivatives. Also we did not have to assume that the errors stay small and
Gaussian distributed throughout the analysis. This bootstrap method allows us
to obtain error estimates for any type of analysis by performing the whole
analysis several times.

A visualization of the bootstrap samples is given by a \enquote{band} of values
along the optimal fit line. This band marks one standard deviation of the
distribution of fit functions at each point. A wide band means a high
uncertainty in the fitted function. For the Gaussian functions, the band is so
narrow that one cannot see it. In the parameter table
(Table~\ref{tab:time_gauge_param}) one can see that the relative error is in
the order of \num{e-4} and therefore no visible uncertainty.

One plots delay time against channel number (see Figure~\ref{fig:time_gauge})
and obtains a linear correlation. Since we do not need absolute time values,
just time differences, only the slope of the fit is of interest for us. It is
\SI{<< time_gauge_slope >>}{\milli\second} per channel. Here the error
estimation is performed with the first order Jackknife method. We perform a
linear fit with only six points. Taking one point away might change the outcome
of the fit if that point was an outlier. In the Jackknife procedure, we repeat
the curve fitting with a single point excluded in one of the fits. Iterating
through the points, we perform six fits. Then again we can take mean and
standard deviation of our fit parameters. The distribution of fit curves is so
narrow that it is hidden below the solid red line. We will see a broad band
later on in a different plot.

\begin{figure}
        \centering
        \includegraphics{time_gauge}
        \caption{%
                Time calibration of \textsc{mca}-channels with linear fit.
        }
        \label{fig:time_gauge}
\end{figure}

The last prompt curve has a standard deviation of \num{<< width_6 >>}. From
this we get $\textsc{fwhm} = \num{<< FWHM_6 >>}$. With the slope of the time
calibration fit we get a resolution of \SI{<< time_resolution
>>}{\nano\second}.

\section{Adjusting SCA windows}

After the time calibration, we have to set up the fast-slow coincidence
circuit. For this we set the \textsc{sca} window on the start-side
to the \SI{1275}{\kilo\electronvolt} line of the spectrum and check if the
two \textsc{sca}s as well as the \textsc{tac} and the coincidence unit have
a nice overlap. This is shown in Figure~\ref{fig:7-sca-coincidence-1275} and
Figure~\ref{fig:8-tac-and-coincidence-1275}, respectively. We see that the
overlap is sufficient.

\begin{figure}
    \centering
    \begin{subfigure}[c]{0.48\linewidth}
        \centering
        \includegraphics[width=\linewidth]{br-7-sca-coincidence-1275}
        \caption{%
                Left and right \textsc{sca}.
        }
        \label{fig:7-sca-coincidence-1275}
    \end{subfigure}
    \hfill
    \begin{subfigure}[c]{0.48\linewidth}
        \centering
        \includegraphics[width=\linewidth]{br-8-tac-and-coincidence-1275}
        \caption{%
                \textsc{Sca} and \textsc{tac}.
        }
        \label{fig:8-tac-and-coincidence-1275}
    \end{subfigure}
    \caption{%
        Coincidence check between different elements of the circuit.
    }
    \label{fig:}
\end{figure}

\section{Indium sample}

An indium sample is placed inside an aluminum block, which is mounted on a
soldering iron. This allows us to control the sample's temperature with a
potentiometer. A thermometer enables us to read off the actual temperature. Now
the sample is placed between the detectors which have to be shielded against
the heat by two aluminum plates. We have to ensure that there is no contact to
the heat shielding to prevent heat conduction. 

\subsection{Lifetime spectrum}

The first measurement is at room temperature. We let the fast-slow circuit
select creation and annihilation events using the decay of exited neon and
positron annihilation radiation, respectively. The lifetimes are recorded using
the \textsc{mca}. After about \SI{30}{\minute} we stop the event recording save
the data to disk. Then We turn up the temperature and wait for the temperature
to stabilize. We repeat this until we have eight data sets. For every
measurement we note the temperature in the beginning and at the end of the
acquisition. Assuming a slow walk in the temperature, the mean and difference
of the two temperatures will provide average value and an error estimate.

In the theory section, we have mentioned the model for the lifetime spectrum in
\eqref{eq:lifetime-fit}. We fit this model to each of the lifetime spectra
recorded. The measurement at room temperature together with a fit function is
shown in Figure~\ref{fig:lifetime-295K}. For the other measurements see
Figures~\ref{fig:lifetime-324K} to \ref{fig:lifetime-398K} in the appendix.

\begin{figure}
    \centering
    \includegraphics{lifetime-295K}
    \caption{%
        Lifetime spectrum of indium at room temperature.
    }
    \label{fig:lifetime-295K}
\end{figure}

From the eight fits at different temperature we obtain a set of lifetimes. The
lifetimes $\tau_0$ and $\tau_\mathrm t$ are direct parameters. The
$\tau_\mathrm f$ and $\bar \tau$ are derived using the relations quoted in our
theoretical introduction. All lifetimes are listed in Table~\ref{tab:taus}.
Also they are plotted in Figure~\ref{fig:taus}. Error estimation is done with
the bootstrap method.

\begin{table}
    \centering
    \begin{tabular}{SSSSS}
        \toprule
        {$T/\si{\kelvin}$}
        & {$\tau_0 / \si{\nano\second}$}
        & {$\tau_\mathrm t / \si{\nano\second}$}
        & {$\tau_\mathrm f / \si{\nano\second}$}
        & {$\bar \tau / \si{\nano\second}$}
        \\
        \midrule
        %< for row in taus_table >%
        << ' & '.join(row) >> \\
        %< endfor >%
        \bottomrule
    \end{tabular}
    \caption{%
        Various lifetimes derived from the spectra. This data is visualized in
        Figure~\ref{fig:taus}.
    }
    \label{tab:taus}
\end{table}

\begin{figure}
    \centering
    \includegraphics{taus}
    \caption{%
        Compilation of the various lifetimes derived from the spectra. Values
        are enumerated in Table~\ref{tab:taus}.
    }
    \label{fig:taus}
\end{figure}

We would expect $\bar\tau$ to have a strong temperature dependence. However, it
is mostly constant. This should not be the case as this would imply a near
constant density of vacancies throughout the temperature range. The mixing of
the two components in $\bar\tau$ is mediated by the ratios of intensities $A_0$
and $A_\mathrm t$. These quantities are also fit parameters and listed in
Table~\ref{tab:intensities}.

\begin{table}
    \centering
    \begin{tabular}{SSS}
        \toprule
        {$T/\si{\kelvin}$}
        & {$A_0^\text{rel}$}
        & {$A_\mathrm t^\text{rel}$}
        \\
        \midrule
        %< for row in intensities_table >%
        << ' & '.join(row) >> \\
        %< endfor >%
        \bottomrule
    \end{tabular}
    \caption{%
        Relative intensities of the non-trapped ($A_0$) and trapped ($A_\mathrm
        t$) parts. These values are plotted in Figure~\ref{fig:intensities}.
    }
    \label{tab:intensities}
\end{table}

The plot in Figure~\ref{fig:intensities} shows that the two intensities depend
only lightly on the temperature. They do not cross over, therefore $\bar\tau$
has only a weak temperature dependence. Our expectation was to find a strong
temperature dependence in $\bar\tau$ which would have a sigmoidal shape. The
two asymptotes would be $\tau_0$ and $\tau_\mathrm t$. Figure~\ref{fig:s_curve}
shows $\bar\tau$ against the temperature. It only varies slightly. Within the
errors of the data points, fitting a constant function would be well justified.

We have tried to fit the model Equation~\eqref{eq:temp-model} to the data. With
out data, it was so unstable that the fit did not converge even after tweaking
the initial parameters carefully. We therefore cannot produce a fit here for
that data. Luckily there is another way to obtain the vacancy formation
enthalpy; we will turn to that next.

\begin{figure}
    \centering
    \includegraphics{intensities}
    \caption{%
        Relative intensities of the non-trapped ($A_0$, upper curve) and
        trapped ($A_\mathrm t$, lower curve) parts. The values are listed in
        Table~\ref{tab:intensities}.
    }
    \label{fig:intensities}
\end{figure}

\begin{figure}
    \centering
    \includegraphics{s_curve}
    \caption{%
        $\bar\tau$ against the temperature. One would expect to see an
        sigmoidal shape here.
    }
    \label{fig:s_curve}
\end{figure}

\subsection{Vacancy formation enthalpy}

The vacancy formation enthalpy~$H_\mathrm t$ is the energy needed for form
another vacancy. We can obtain it from the Arrhenius plot using
Equation~\ref{eq:arrhenius}. First we need to convert the data that we have
into the required form. The transformed data is listed in
Table~\ref{tab:arrhenius}.

\begin{table}
    \centering
    \begin{tabular}{SS}
        \toprule
        {$\si{\kelvin} / T$}
        % TODO Which unit?
        & {$\sigma_\mathrm t C_\mathrm t(T)$}
        \\
        \midrule
        %< for row in arrhenius_table >%
        << ' & '.join(row) >> \\
        %< endfor >%
        \bottomrule
    \end{tabular}
    \caption{%
        Converted values for the Arrhenius plot shown in
        Figure~\ref{fig:arrhenius}.
    }
    \label{tab:arrhenius}
\end{table}

From the data we create a plot (Figure~\ref{fig:arrhenius}) and fit an
exponential decay. An alternative is a logarithmic plot with a linear fit. The
solid red line shows the optimal fitted curve. Error estimation is done by a
combination of bootstrap and Jackknife methods. From the previous steps, we
have a whole distribution of lifetimes. Converting them will give us
distributions of values for $\sigma_\mathrm t C_\mathrm t(T)$. Performing the
curve fit once for every distribution gives us a distribution of fit
parameters. As there are several outliers, the Jackknife procedure also
accounts for the uncertainty due to their presence.

\begin{figure}
    \centering
    \includegraphics{arrhenius}
    \caption{%
        Arrhenius plot with a linear ordinate. The band illustrates the
        possible exponential decay fits. The values are listed in
        Table~\ref{tab:arrhenius}.
    }
    \label{fig:arrhenius}
\end{figure}

From all those fits, we obtain a vacancy formation enthalpy of
\[
    H_\mathrm t = \SI{<< Ht_eV >>}{\electronvolt} \,.
\]

\section{Acrylic glass sample}

After we take all temperature dependent measurements with the indium sample, we
place an acrylic glass sample between the detectors. Inside the glass, there is
$\mathrm{^{22}Na}$ dust. We measure at room temperature over night.

Figure~\ref{fig:acrylic} shows the resulting spectrum we got the next morning.
As done before for the indium, we fit the same two-decay model to our data. In
contrast to the indium spectra, this fit turned out to be rather unstable. We
did a few things to get it under control. First we fitted a simple exponential
decay to the straight parts of the falling flank. This gave us a first estimate
of the lifetimes for the positronium. We obtained
\[
    \tau_0 = \SI{<< acryl_tau_0_lin >>}{\nano\second}
    \eqnsep
    \tau_\mathrm t = \SI{<< acryl_tau_t_lin >>}{\nano\second} \,.
\]
Choosing fixed limits in the time for the fit, we introduce a systematic error.
The proper treatment would be a $\chi^2$-weighted histogram of fit parameters
to control the statistical error. We do not need this treatment here, we just
want to use those values as starting values for the unstable fit. Using those
lifetimes as input for the full fit model, we obtained the lifetimes
\[
    \tau_0 = \SI{<< acryl_tau_0 >>}{\nano\second}
    \eqnsep
    \tau_\mathrm t = \SI{<< acryl_tau_t >>}{\nano\second}
    \eqnsep
    \tau_\mathrm f = \SI{<< acryl_tau_f >>}{\nano\second}
    \eqnsep
    \bar\tau = \SI{<< acryl_tau_bar >>}{\nano\second}
    \,.
\]

The fit was done in the vertical logarithmic scale, the internally used
$\chi^2$ the logarithmic counts to compute the likelihood of the fit. Bins with
zero counts were excluded as $\log(0)$ does not give a sensible value. Points
before \SI{10}{\nano\second} were also excluded as the model significantly
deviates from our data. The actual physic happens in the falling flank,
therefore we are allowed to discard the rising flank. Error estimation was done
using bootstrap alone. With over 8000 points, a first order Jackknife would not
alter anything.

Figure~\ref{fig:acrylic} shows the data and the fit curve.
Figure~\ref{fig:acrylic-zoom} shows the data and the various fit lines in more
detail.

\begin{figure}
    \centering
    \includegraphics{acrylic}
    \caption{%
        Overnight measurement with acrylic glass. Shown in black are the counts
        observed with the \textsc{mca}. The gray line is a smoothed version of
        the data (convolution with Gauss kernel) in order to help estimate the
        mean value in the jitter. Our fitted model is shown in red. The two
        dashed blue line segments are a decaying exponential fit to the
        straight sections of the curve. Their slope is used as a starting value
        for the red fit line.
    }
    \label{fig:acrylic}
\end{figure}

\begin{figure}
    \centering
    \includegraphics{acrylic-zoom}
    \caption{%
        Zoom-in of Figure~\ref{fig:acrylic}.
    }
    \label{fig:acrylic-zoom}
\end{figure}

% TODO Which lifetimes are accessible with this setup?

% TODO Some hand waiving or so …

\chapter{Conclusion}

We were able to derive the vacancy formation enthalpy to be
\[
    H_\mathrm t = \SI{<< Ht_eV >>}{\electronvolt} \,.
\]
The result found by \textcite[(7a)]{Weiler/Vacancy_formation} is \SI{0.54 +-
0.03}{\electronvolt}. They also feature a table which contains results from
other groups. In all cases the values are around \SI{0.5}{\electronvolt}
\parencite[Table~1]{Weiler/Vacancy_formation}. We must therefore conclude that
our value is \emph{way} off the expected value. Our estimation clearly states
that the two results are incompatible with each other.

% TODO Possible sources for errors.

For the acrylic glass, we found the lifetimes of the positronium to be
\[
    \tau_0 = \SI{<< acryl_tau_0 >>}{\nano\second}
    \eqnsep
    \tau_\mathrm t = \SI{<< acryl_tau_t >>}{\nano\second}
    \eqnsep
    \tau_\mathrm f = \SI{<< acryl_tau_f >>}{\nano\second}
    \eqnsep
    \bar\tau = \SI{<< acryl_tau_bar >>}{\nano\second}
    \,.
\]

% TODO Comparison with literature values.

\begin{appendix}

    \chapter{Lifetime spectra}

    %< for temp in temps_int[1:] >%
    \begin{figure}[htbp]
        \centering
        \includegraphics{lifetime-<< temp >>K}
        \caption{%
            Lifetime spectrum of indium at \SI{<< temp >>}{\kelvin}.
        }
        \label{fig:lifetime-<< temp >>K}
    \end{figure}
    %< endfor >%
\end{appendix}

\end{document}

% vim: spell spelllang=en_us tw=79
